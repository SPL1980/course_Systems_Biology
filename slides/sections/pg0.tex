\begin{frame}
  \frametitle{自我介绍}
    \begin{description}
      \item[姓\qquad 名]伊现富(Yi Xianfu)
      \item[本\qquad 科]山东大学
      \item[硕\qquad 博]中国科学院
      \item[工作邮箱]\alert{yixfbio@gmail.com}
      \item[生活邮箱]yixf1986@gmail.com
      \item[手\qquad 机]\alert{156\ 2061\ 0763}
      % \item[个人博客]\href{http://yixf.name}{http://yixf.name}
      \item[@GitHub]\alert{\href{https://github.com/Yixf-Education}{https://github.com/Yixf-Education}}
      \item[网络昵称]yixf, Yixf
    \end{description}
\end{frame}

\begin{frame}
  \frametitle{授课资料}
  \begin{figure}
    \centering
    \includegraphics[width=6.5cm]{qr.png}
  \end{figure}
  \begin{center}
  \href{https://github.com/Yixf-Education/course_Systems_Biology}{https://github.com/Yixf-Education/course\_Systems\_Biology}
  \end{center}
\end{frame}

\begin{frame}
  \frametitle{课程安排}
  \begin{table}
    \centering
    \rowcolors[]{1}{blue!20}{blue!10}
    \begin{tabular}{clccc}
      \hline
      \rowcolor{blue!50}日期 & 授课内容 & 学时 & 授课方式 & 授课教师\\
      \hline
      11.19 & 高通量测序技术及数据分析 & 3 & 理论授课 & 伊现富\\
      11.26 & \tiny{新型抗炎纳米粒对急性肺损伤的治疗效用与机制} & 3 & 理论授课 & 杨红\\
      12.03 & 系统生物学的建模和仿真 & 3 & 理论授课 & 王举\\
      12.10 & 生物大分子的药物肿瘤治疗 & 3 & 理论授课 & 高秀军\\
      12.17 & 蛋白质分子模拟 & 3 & 理论授课 & 张涛\\
      12.24 & 系统生物学的相关应用 & 3 & 课堂讨论 & 张涛\\
      \hline
    \end{tabular}
  \end{table}
  \vspace{-1em}
  \begin{block}{课程考核(\alert{严禁抄袭})}
    通过查阅文献,撰写研究进展报告。
    \begin{itemize}
      \item 小论文:课程主题相关/研究方向相关/自己兴趣所在
      \item 课堂讨论:课程主题相关,幻灯片,讲解5~8分钟
    \end{itemize}
  \end{block}
\end{frame}
